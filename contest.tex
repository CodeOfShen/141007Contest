\documentclass{article}
\usepackage{geometry}
\geometry{left=2.5cm,right=2.5cm,top=2.5cm,bottom=2.5cm}
%---------------------------------------
% 必要库支持
%---------------------------------------
\usepackage{amsmath}
\usepackage{xcolor}
\usepackage{multirow}
\usepackage{longtable}
\renewcommand{\sfdefault}{phv}
\usepackage{listings}
\usepackage{courier}
\lstset{
    basicstyle=\footnotesize\ttfamily, % Standardschrift
    numbers=left,          % Ort der Zeilennummern
    numberstyle=\tiny,     % Stil der Zeilennummern
    numbersep=5pt,         % Abstand der Nummern zum Text
    tabsize=2,        % Groesse von Tabs
    extendedchars=true,    %
    breaklines=true,       % Zeilen werden Umgebrochen
    keywordstyle=\color{blue},
    numberstyle=\color[RGB]{0,192,192},
    commentstyle=\color[RGB]{0,96,96},      
    stringstyle=\color[RGB]{128,0,0}\ttfamily, % Farbe der String
    showspaces=false,      % Leerzeichen anzeigen ?
    showtabs=false,        % Tabs anzeigen ?
    xleftmargin=17pt,
    framexleftmargin=17pt,
    framexrightmargin=5pt,
    framexbottommargin=4pt,
    showstringspaces=false
}
\renewcommand{\lstlistingname}{CODE}
\lstloadlanguages{% Check Dokumentation for further languages ...
    %[Visual]Basic
    %Pascal
    %C
    C++
    %XML
    %HTML
    %Java
}
  \usepackage{caption}
\DeclareCaptionFont{white}{\color{white}}
\DeclareCaptionFormat{listing}{\colorbox[cmyk]{0.43, 0.35, 0.35,0.01}{\parbox{\textwidth}{#1#2#3}}}
\captionsetup[lstlisting]{format=listing,justification=raggedright,labelfont=white,textfont=white, singlelinecheck=false, margin=0pt, font={sf,bf,footnotesize}
}

%---------------------------------------
% 设置中文字体
%---------------------------------------
\usepackage[SlantFont,BoldFont,CJKchecksingle,CJKnumber]{xeCJK}
    \newcommand\fontnamehei{Adobe Heiti Std}
    \newcommand\fontnamesong{Adobe Song Std}
    \newcommand\fontnamekai{Adobe Kaiti Std} % 楷体
    \newcommand\fontnameyahei{Yahei}
    \newcommand\fontnameCode{YaHei Consolas Hybrid}
    
\defaultfontfeatures{Mapping=tex-text}
\setCJKmainfont[BoldFont=\fontnamehei,ItalicFont=\fontnamekai]{\fontnamesong}    
\setCJKmonofont{\fontnameCode}
\setCJKsansfont[BoldFont=\fontnamehei]{\fontnameyahei}
\setmonofont{Source Code Pro}
\XeTeXlinebreaklocale "zh"       %使用中文的换行风格
\XeTeXlinebreakskip = 0pt plus 1pt    %调整换行逻辑的弹性大小

%---------------------------------------
% 设置标题格式
%---------------------------------------
\usepackage[center]{titlesec}
\titleformat{\sectio}{\centering\Huge\bfseries}{}{1em}{}
\titleformat{\subsection}{\Large\bfseries}{}{1em}{}
%---------------------------------------
% 设置标题、作者、日期
%---------------------------------------
\title{十一测试赛}
\author{Shangkun Shen}
\date{\today}
\newcommand\artversion{1.0}
\newcommand\artdate{Oct 7th, 2014}

%---------------------------------------
% 设置页眉页脚
%---------------------------------------
\usepackage{fancyhdr}
\pagestyle{fancy}
\lhead{十一测试赛}
\rhead{\artdate}
\chead{} 
\lfoot{--Stay Hungry Stay Foolish--}
\cfoot{}
\rfoot{\thepage} 
\renewcommand{\headrulewidth}{0.4pt} 
\renewcommand{\headwidth}{\textwidth} 
\renewcommand{\footrulewidth}{0pt}

%---------------------------------------
% 文章内容
%---------------------------------------
\begin{document}
%---------------------------------------
% 添加目录
%---------------------------------------
\maketitle
\begin{longtable}{|c|l|}
 \hline 比赛日期 & \artdate \\
 \hline 赛制 & OI赛制 \\
 \hline 比赛时间 & 5小时 \\
 \hline 题目数量 & 5题 \\
 \hline 分值设定 & 每题100分,满分500分 \\
 \hline
\end{longtable}
\thispagestyle{fancy}
\begin{itemize}
 \item  本次测试比赛采用OI赛制,参赛选手可以使用Pascal, C/C++, 
        Java来完成比赛题目。最后比赛总分得分最高者即为冠军,其他
        名次依得分顺延。
 \item  所有的数据读入读出采用文件读入读出形式,即要求必须使用文
        件输入输出,并且文件名必须符合题目要求,否则将不会得到成
        绩。
 \item  使用Java的同学,主类名为Main,代码的文件名根据题目中的说
        明来正确命名即可。
 \item  题目的时间限制已经按照2倍标程(C/C++)进行了放宽处理,而
        Java按照3倍时限进行放宽处理。
\end{itemize}
\clearpage
%---------------------------------------
% 文章正文
%---------------------------------------
\section*{A.丹丹的生日派对}
%---------------------------------------
\begin{longtable}{|c|l|}
 \hline \bfseries{时间限制} & 1s \\
 \hline \bfseries{空间限制} & 64MB \\
 \hline \bfseries{源程序名} & queue.\{c,cpp,java\} \\
 \hline \bfseries{输入文件} & queue.in \\
 \hline \bfseries{输出文件} & queue.out \\
 \hline 
\end{longtable}

\subsection*{问题描述}
丹丹的生日就快到了,她的好伙伴们一直在帮她策划一个新的活动,没
错,就是举办一个派对!这样她的好朋友们都可以过来一起愉快的玩耍,
并且能够分享她的喜悦了!

但是令丹丹意外的是,来的人实在是太多了,太多了!多得让丹丹的头
都大了。在这群排队的人中,有男生,有女生,还有熊孩子(既然叫熊
孩子,自然就不考虑它们这一人群的性别了)。而且丹丹了解到:如果
三个女孩排在一起,她们就会互相攀比;如果三个男孩排在一起,他们
就会喋喋不休;如果三个熊孩子排在一起,哦我的上帝!天都要塌下来
了!一旦出现这种情况,丹丹就会为这些吵闹的人感到惋惜而无心过生
日。而她的好伙伴们则希望丹丹总是开开心心,所以他们接着去分析这
个问题。

他们注意到,除了上述所讲的三种情况之外,还有不少男生女生是借丹
丹的生日派对之便为趁机搭讪而来的。所以如果派对的过程中,男生和
女生之间夹有一个熊孩子,这样这对少男少女就会十分无奈,丹丹也会
为此惋惜而变得郁郁寡欢。另外,如果男生或者女生被两个熊孩子加在
中间,熊孩子迟早会把那个倒霉的人惹烦。这样一来,丹丹会觉得她办
了一个失败的生日派对而郁郁寡欢。

现在丹丹的好伙伴们跑来询问你,针对一个排了$N$人的队伍,到底有多
少种让丹丹满意的排列方案?由于答案可能会很庞大,所以他们麻烦你
将答案对1,000,000,007取模后,再告诉他们。

\subsection*{输入数据}
有十组测试数据。每一组的输入数据格式相同。

第一行是一个整数$T$,表示有$T$组测试样例。

之后的$T$行中,每一行有一个数字$N$,表示这个队伍的总人数。

\subsection*{输出数据}
针对每一个测试样例,输出一行结果。

结果的格式为{\tt "Case \#No: Ans"},
No表示第No组样例,Ans表示该组样例的答案。

具体的请参见输出样例。

\subsection*{测试样例}
\begin{flushleft}
\begin{tabular}{|p{6cm}|p{6cm}|}
 \hline \bfseries{样例输入} & \bfseries{样例输出} \\
 \hline 
    3 & {\tt Case \#1: 20} \\
    3 & {\tt Case \#2: 46} \\
    4 & {\tt Case \#3: 435170} \\
    15 & \\
 \hline 
\end{tabular}
\end{flushleft}

\subsection*{数据规模}
对于所有的数据,$T$满足$1 \leq T \leq 1500$。

对于$30\%$的数据,每组3分,保证$N$满足$0 \leq N \leq 10^{4}$。


对于$30\%$的数据,每组7分,保证$N$满足$0 \leq N \leq 10^{7}$。


对于$20\%$的数据,每组15分,保证$N$满足$0 \leq N \leq 10^{9}$。


对于$20\%$的数据,每组20分,保证$N$满足$0 \leq N \leq 10^{11}$。

\clearpage
%---------------------------------------

\section*{B.丹丹的数学问题}
%---------------------------------------
\begin{longtable}{|c|l|}
 \hline \bfseries{时间限制} & 1s \\
 \hline \bfseries{空间限制} & 64MB \\
 \hline \bfseries{源程序名} & math.\{c,cpp,java\} \\
 \hline \bfseries{输入文件} & math.in \\
 \hline \bfseries{输出文件} & math.out \\
 \hline 
\end{longtable}

\subsection*{问题描述}
数学作业总是最烦人的一项,不论何时,不论何地。丹丹的数学作业还
没写完,但是她又想在生日派对上好好地轻松一天,所以她把她的数学
作业就交给你了。

其实作业只有一道题目,讲的是一个正整数数列$1, 2, \cdots , n$,
每一次都可以选取任意多个数列中的数据进行一次操作,让这些被选中
的数字统一减去一个相同的正整数,但是不能减成负数。目标是将整个
数列的数通过上述操作全部变成0。求最少的操作次数。

“这个问题太简单了!”你这样想着,“告诉我这个数列$n$的大小,我来
告诉你答案。”

\subsection*{输入数据}
有十组测试数据。每一组的输入数据格式相同。

第一行是一个整数$T$,表示有$T$组测试样例。

之后的$T$行中,每一行有一个数字$N$,表示这个数列的长度。

\subsection*{输出数据}
针对每一个测试样例,输出一行结果。

结果的格式为{\tt "Case \#No: Ans"},
No表示第No组样例,Ans表示该组样例的答案。

具体的请参见输出样例。

\subsection*{测试样例}
\begin{flushleft}
\begin{tabular}{|p{6cm}|p{6cm}|}
 \hline \bfseries{样例输入} & \bfseries{样例输出} \\
 \hline 
    3 & {\tt Case \#1: 1} \\
    1 & {\tt Case \#2: 2} \\
    2 & {\tt Case \#3: 2} \\
    3 & \\
 \hline 
\end{tabular}
\end{flushleft}

\subsection*{数据规模}
对于$20\%$的数据,每组2分,保证$T$满足$1 \leq T \leq 10^{2}$,
保证$N$满足$1 \leq N \leq 10^{3}$。


对于$30\%$的数据,每组4分,保证$T$满足$1 \leq T \leq 10^{3}$,
保证$N$满足$1 \leq N \leq 10^{6}$。


对于$20\%$的数据,每组15分,保证$T$满足$1 \leq T \leq 10^{4}$,
保证$N$满足$1 \leq N \leq 10^{8}$。


对于$30\%$的数据,每组18分,保证$T$满足$1 \leq T \leq 10^{6}$,
保证$N$满足$1 \leq N \leq 10^{9}$。

\clearpage
%---------------------------------------

\section*{C.丹丹的生日礼物}
%---------------------------------------
\begin{longtable}{|c|l|}
 \hline \bfseries{时间限制} & 3s \\
 \hline \bfseries{空间限制} & 256MB \\
 \hline \bfseries{源程序名} & prize.\{c,cpp,java\} \\
 \hline \bfseries{输入文件} & prize.in \\
 \hline \bfseries{输出文件} & prize.out \\
 \hline 
\end{longtable}

\subsection*{问题描述}
丹丹的好伙伴们为丹丹提前准备好了生日礼物,当丹丹看到这么多的礼
物的时候,不禁感叹:“太多了!你们太好了!”

丹丹不想把所有的礼物全部拿走,因为在这些礼物中,有的太贵重了,
她不想独占所有的礼物,她想把这份开心分享给他的好伙伴们。所以她
决定只拿走总价值为K的礼物。

丹丹的好伙伴们总共为丹丹准备了N种不同的礼物,每种礼物有3种不同
的样式。对每一种不同的礼物,丹丹都可以选取任意件数的该种礼物,
只要拿走的数量不超过这种礼物的总数3件就行。

现在丹丹想知道,她到底能不能拿走总价值恰好为K的礼物。

\subsection*{输入数据}
有十组测试数据。每一组的输入数据格式相同。

第一行是一个整数$T$,表示有$T$组测试样例。

每组数据的第一行是两个整数$N$和$K$,第二行会有$N$个数,分别是
送来的礼物的价值$w_i$。

\subsection*{输出数据}
针对每一个测试样例,输出一行结果。

结果的格式为{\tt "Case \#No: Ans"},
No表示第No组样例,Ans表示该组样例的答案。如果可以取走礼物输出
{\tt yes},否则输出{\tt no}。

具体的请参见输出样例。

\subsection*{测试样例}
\begin{flushleft}
\begin{tabular}{|p{6cm}|p{6cm}|}
 \hline \bfseries{样例输入} & \bfseries{样例输出} \\
 \hline 
    3 & {\tt Case \#1: yes} \\
    2\ 5 & {\tt Case \#2: no} \\
    1\ 2 & {\tt Case \#3: yes} \\
    2\ 10 & \\
    1\ 2 & \\
    3\ 10 & \\
    1\ 3\ 5 & \\
 \hline 
\end{tabular}
\end{flushleft}

\subsection*{数据规模}
对于所有的数据,$T$满足$1 \leq T \leq 20$。

对于$10\%$的数据,每组2分,
保证$N$满足$1 \leq N \leq 5$,
$K$满足$1 \leq K \leq 10^{9}$,
$w_i$满足$1 \leq w_i \leq 10^{3}$。

对于$20\%$的数据,每组4分,
保证$N$满足$1 \leq N \leq 8$,
$K$满足$1 \leq K \leq 10^{9}$,
$w_i$满足$1 \leq w_i \leq 10^{4}$。

对于$30\%$的数据,每组10分,
保证$N$满足$1 \leq N \leq 12$,
$K$满足$1 \leq K \leq 10^{9}$,
$w_i$满足$1 \leq w_i \leq 10^{5}$。

对于$40\%$的数据,每组15分,
保证$N$满足$1 \leq N \leq 16$,
$K$满足$1 \leq K \leq 10^{9}$,
$w_i$满足$1 \leq w_i \leq 10^{6}$。

\clearpage
%---------------------------------------

\section*{D.丹丹的路边向导}
%---------------------------------------
\begin{longtable}{|c|l|}
 \hline \bfseries{时间限制} & 3s \\
 \hline \bfseries{空间限制} & 64MB \\
 \hline \bfseries{源程序名} & guide.\{c,cpp,java\} \\
 \hline \bfseries{输入文件} & guide.in \\
 \hline \bfseries{输出文件} & guide.out \\
 \hline 
\end{longtable}

\subsection*{问题描述}
为了方便来参加丹丹的生日派对的大家准确的找到丹丹的家,丹丹和她
的好伙伴们想到了一个方法,那就是在路口放置上路标,这样大家就能
找到路了。

但是在每一个路口都放置上路标显然是不可能的,丹丹不希望她的好伙
伴们为此太费心费力,所以丹丹的好伙伴们把这个问题再次简化。由于
大家都是来丹丹的家里参加她的生日派对,所以相当于从丹丹的家发散
出了一个道路网。而且,如果一个路口上有一个路标,与这个路口相连
的道路都能看到路标上的信息。为了让所有的人能够正确的找到丹丹的
家,丹丹希望能够用最少的路标,来保证大家都能找到丹丹的家。

\subsection*{输入数据}
有十组测试数据。每一组的输入数据格式相同。

第一行是一个整数$T$,表示有$T$组测试样例。

每组数据的第一行是一个整数$N$,表示有$N$个路口。随后的$N$行中,
每一行都有至少两个整数。第一个整数$i$表示这个路口的标号(从0
开始标记,到$N - 1$为止)。第二个整数$d$表示这个路口所连接的新
的路口数量。后面的$d$个整数分别表示这个路口所连接的新的路口的
标号。如果第二个整数$d = 0$,你可以认为这是路口恰好是一个要来
参加派对的人的住址。

\subsection*{输出数据}
针对每一个测试样例,输出一行结果。

结果的格式为{\tt "Case \#No: Ans"},
No表示第No组样例,Ans表示该组样例的答案,即最少的路标放置数量。

具体的请参见输出样例。

\subsection*{测试样例}
\begin{flushleft}
\begin{tabular}{|p{6cm}|p{6cm}|}
 \hline \bfseries{样例输入} & \bfseries{样例输出} \\
 \hline 
    2 & {\tt Case \#1: 1} \\
    4 & {\tt Case \#2: 2} \\
    0\ 1\ 1 & \\
    1\ 2\ 2\ 3 & \\
    2\ 0 & \\
    3\ 0 & \\
    5 & \\
    3\ 3\ 1\ 4\ 2 & \\
    1\ 1\ 0 & \\
    2\ 0 & \\
    0\ 0 & \\
    4\ 0 & \\
 \hline 
\end{tabular}
\end{flushleft}

\subsection*{样例解释}
对于第1组样例,在1号路口放置路标即可。

对于第2组样例,可以在1号路口和3号路口放置路标,或者在0号和3号
放置路标。以上两种方案均可。

\subsection*{数据规模}
对于所有的数据,$T$满足$50 \leq T \leq 100$。

对于$10\%$的数据,每组2分,
保证$N$满足$1 \leq N \leq 100$。

对于$20\%$的数据,每组4分,
保证$N$满足$1 \leq N \leq 500$。

对于$30\%$的数据,每组10分,
保证$N$满足$1 \leq N \leq 4000$。

对于$40\%$的数据,每组15分,
保证$N$满足$1 \leq N \leq 8000$。

\clearpage
%---------------------------------------

\section*{E.丹丹的合影留念}
%---------------------------------------
\begin{longtable}{|c|l|}
 \hline \bfseries{时间限制} & 1.5s \\
 \hline \bfseries{空间限制} & 64MB \\
 \hline \bfseries{源程序名} & photo.\{c,cpp,java\} \\
 \hline \bfseries{输入文件} & photo.in \\
 \hline \bfseries{输出文件} & photo.out \\
 \hline 
\end{longtable}

\subsection*{问题描述}
丹丹的生日派对就要结束了,“来的人真多”,丹丹这样想着,“大家一
起合个影留个念吧!”

但是事情总不是那么轻轻松松简简单单,丹丹希望每个人在合影时都有
一个位置,不至于到最后照片上只有半张脸。在丹丹和她的好伙伴们讨
论之后,他们决定先把人一个一个的站好,排列成一个$M \times N$矩
形,即有M行,每行N个人的矩形方阵。排好队之后再拍照留念。

这个时候问题来了(不是挖掘机我谢谢你的自动脑补=。=),人数不会
总是那么巧合正好可以按照这个方法排列成一个矩形方阵。所以他们想
到了一些简单的挽救方案。允许中间留空,但是为了整体的美观,不允
许在整个矩形的边界出现空白,也不允许一个空白的周围(前、后、左、
右、左前、右前、左后、右后)出现空白。那么如果现在告诉你当前准
备排列的行数M与每行的人数N,以及总的人数K,能不能告诉丹丹和她
的好伙伴们能不能排列出一个合适的矩形,从而让她们成功合影。

\subsection*{输入数据}
有十组测试数据。每一组的输入数据格式相同。

第一行是一个整数$T$,表示有$T$组测试样例。

之后的$T$行中,每一行有三个数字$M,\ N,\ K$,分别表示矩形方阵的
行数,每行的人数,和总人数。

\subsection*{输出数据}
针对每一个测试样例,输出一行结果。

结果的格式为{\tt "Case \#No: Ans"},
No表示第No组样例,Ans表示该组样例的答案。如果可以排列成$M \times N$
的矩形方阵,输出{\tt yes},否则输出{\tt no}。

具体的请参见输出样例。

\subsection*{测试样例}
\begin{flushleft}
\begin{tabular}{|p{6cm}|p{6cm}|}
 \hline \bfseries{样例输入} & \bfseries{样例输出} \\
 \hline 
    3 & {\tt Case \#1: yes} \\
    3\ 3\ 9 & {\tt Case \#2: yes} \\
    3\ 3\ 8 & {\tt Case \#3: no} \\
    3\ 3\ 3 & \\
 \hline 
\end{tabular}
\end{flushleft}

\subsection*{数据规模}
对于所有的数据,$T$满足$150 \leq T \leq 200$。

对于$10\%$的数据,每组2分,
保证$N,\ M$满足$1 \leq N,\ M \leq 10^{6}$,
$K$满足$1 \leq K \leq 10^{13}$


对于$20\%$的数据,每组4分,
保证$N,\ M$满足$1 \leq N,\ M \leq 10^{12}$,
$K$满足$1 \leq K \leq 10^{25}$


对于$30\%$的数据,每组10分,
保证$N,\ M$满足$1 \leq N,\ M \leq 10^{1000}$,
$K$满足$1 \leq K \leq 10^{1000}$


对于$40\%$的数据,每组15分,
保证$N,\ M$满足$1 \leq N,\ M \leq 10^{1500}$,
$K$满足$1 \leq K \leq 10^{1500}$

\clearpage
%---------------------------------------



%---------------------------------------
% 文章结束
%---------------------------------------
\end{document}